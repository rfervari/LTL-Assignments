\documentclass{article}


\usepackage[T1]{fontenc}
%\usepackage[latin1]{inputenc}
\usepackage[x11names,dvipsnames]{xcolor}
\usepackage{amsmath}
\usepackage{amssymb}
\usepackage{verbatim}
\usepackage{booktabs}
\usepackage{multicol}
\usepackage{xspace}
\usepackage{underscore}
\usepackage{upgreek}
\usepackage[textsize=scriptsize]{todonotes}
\usepackage{lineno}
%\usepackage{pxfonts}
\usepackage{algorithm}
\usepackage{algpseudocode}
\usepackage{stmaryrd}
\usepackage{amsthm}
\usepackage[draft]{fixme}

\usepackage[font=small,skip=0pt]{caption}

\usepackage{mathabx}
\usepackage{wasysym}
\usepackage{mathtools}
%------------------------------------------------------------------------------------------------

%------------------------------------------------------------------------------------------------

\usepackage{hyperref}

\hypersetup{
  pdfstartview      = {FitH},
  pdfpagelayout     = {OneColumn},
  pdfpagelabels     = true,             % for the reader to display the page number as 'ii (4 of 40)' rather than simply '4 of 40'
  plainpages        = false,            % page anchors with formatted form of the page number (i.e., different anchors for pages 'ii' and '2')
  bookmarksnumbered = {true},
  naturalnames      = {true},
  colorlinks        = {true},           
  linkcolor         = blue,             
  anchorcolor       = red,              % ?
  urlcolor          = NavyBlue,         
  citecolor         = BrickRed          
}

%------------------------------------------------------------------------------------------------

\usepackage{macros}

%------------------------------------------------------------------------------------------------

% \title{Past $\LTL$ with Assignments on Simple Domains}
\title{Temporal Logic with Assignments Modulo Simple Theories}

\author{St\'ephane Demri and Raul Fervari}

\date{ } 

%------------------------------------------------------------------------------------------------

\begin{document}
\maketitle 

\begin{abstract}
  In this paper we investigate the computational complexity and the expressive power of a family of temporal logics featuring simple assignment operators. More precisely, we extend Past $\LTL$ ($\PLTL$) with assigment operators, acting over different domains. First, we study the effect of updating the propositional value of a variable according to the truth value of the truth value of some ``locally visible'' variable. We show that with this simple update, we obtain full $\FOL$ expressivity, and moreover, even in restricted cases, this extension leads to a $\tower$-complete satisfiability problem. Then, we extend $\PLTL$ with concrete domains and operators that update the value of variables ranging over such a domain. TBC...
\end{abstract}

\section{Introduction}

\section{$\PLTL$ with Propositional Assignments}

\input{definitions}

\section{$\PLTL$ with Assignments on Concrete Domains}


\bibliographystyle{plain}
\bibliography{biblio}


\end{document}
