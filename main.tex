\documentclass{article}


\usepackage[T1]{fontenc}
%\usepackage[latin1]{inputenc}
\usepackage[x11names,dvipsnames]{xcolor}
\usepackage{amsmath}
\usepackage{amssymb}
\usepackage{verbatim}
\usepackage{booktabs}
\usepackage{multicol}
\usepackage{xspace}
\usepackage{underscore}
\usepackage{upgreek}
\usepackage[textsize=scriptsize]{todonotes}
\usepackage{lineno}
%\usepackage{pxfonts}
\usepackage{algorithm}
\usepackage{algpseudocode}
\usepackage{stmaryrd}
\usepackage{amsthm}
\usepackage[draft]{fixme}

\usepackage[font=small,skip=0pt]{caption}

\usepackage{mathabx}
\usepackage{wasysym}
\usepackage{mathtools}
%------------------------------------------------------------------------------------------------

%------------------------------------------------------------------------------------------------

\usepackage{hyperref}

\hypersetup{
  pdfstartview      = {FitH},
  pdfpagelayout     = {OneColumn},
  pdfpagelabels     = true,             % for the reader to display the page number as 'ii (4 of 40)' rather than simply '4 of 40'
  plainpages        = false,            % page anchors with formatted form of the page number (i.e., different anchors for pages 'ii' and '2')
  bookmarksnumbered = {true},
  naturalnames      = {true},
  colorlinks        = {true},           
  linkcolor         = blue,             
  anchorcolor       = red,              % ?
  urlcolor          = NavyBlue,         
  citecolor         = BrickRed          
}

%------------------------------------------------------------------------------------------------

\usepackage{macros}

%------------------------------------------------------------------------------------------------

% \title{Past $\LTL$ with Assignments on Simple Domains}
\title{\bf Temporal Logic with Assignments Modulo Theories}

\author{St\'ephane Demri and Raul Fervari}

\date{ } 

%------------------------------------------------------------------------------------------------

\begin{document}
\maketitle 

\begin{abstract}
  In this paper we investigate the computational complexity and the expressive power of a family of temporal logics featuring simple assignment operators. More precisely, we extend Past $\LTL$ ($\PLTL$) with assigment operators, acting over different domains. First, we study the effect of updating the propositional value of a variable according to the truth value of the truth value of some ``locally visible'' variable. We show that with this simple update, we obtain full $\FOL$ expressivity, and moreover, even in restricted cases, this extension leads to a $\tower$-complete satisfiability problem. Then, we extend $\PLTL$ with concrete domains and operators that update the value of variables ranging over such a domain. TBC...
\end{abstract}

\section{Introduction}

\section{$\PLTL$ with Assignments on Concrete Domains}

We start by introducing the logic extending $\LTL$ introduced first by~\cite{Pnueli77}. 


\paragraph{Concrete Domains.} In what follows we will consider a concrete domain $\tup{\adomain,=,<,,(=_{\adatum})_{\adatum\in\adomain}}$, where $\adomain$ is a set of elements, $=,<$ are the correspondent equality and `less than' relation, and $=_\adatum$ is an equality test with respect to the constant $\adatum$.  In what follows we will write just $\adomain$ for $\tup{\adomain,=,<,=_{\adatum}}$ to abbreviate.

Let $\VAR=\{\avar,\avarbis,\avarter,\ldots\}$ be a countably infinite set of variables. A \defstyle{term} $\aterm$ over $\VAR$ is an expression of the form $\Nextn\avar$ or $\Prevn\avar$, where $\avar\in\VAR$ and $\Nextn$ (resp., $\Prevn$) is a (possibly empty) sequence of $n$ symbols $\Next$ (resp., $\Prev$). 
More precisely, the expressions $\Nextn\avar$ and $\Prevn\avar$ are inductively defined as: $\Next^0\avar \egdef \avar$ and $\Prev^0\avar \egdef \avar$; $\Next^{n+1}\avar \egdef \Next\Nextn\avar$ and $\Prev^{n+1}\avar \egdef \Prev\Prevn\avar$. 
An \defstyle{atomic constraint} is an expression of one of the forms below:

\[
    \begin{array}{c@{\quad\quad}c@{\quad\quad}c}
        \aterm = \aterm' & \aterm < \aterm' & {=_{\adatum}}(\aterm) \mbox{ (also written $\aterm=\adatum$)},
    \end{array}
\]
where $\adatum\in\adomain$ and $\aterm,\aterm'$ are terms. Let $\aterm$ be $\Nextn\avar$ or $\Prevn\avar$, we define $\aterm_v\egdef\avar$, and for $i\neq 0$, define $\pos^i(\Nextn\avar)\egdef i+n$ and $\pos^i(\Prevn\avar) \egdef i-n$. 
%Terms are interpreted over valuations $\avaluation: \VAR \to \adomain$. 

\paragraph{Syntax.} Let $\VAR$ be a countable set of variables over the concrete domain $\adomain$. The syntax of $\PLTLA(\adomain)$ is defined by the following grammar:
\[
    \begin{array}{l@{\;}l}
    \aformula,\aformulabis  ::= & \aterm = \aterm' \mid \aterm < \aterm' \mid \aterm = \adatum \mid \neg\aformula \mid \aformula \vee \aformulabis \\
     &  \mid \Next\aformula \mid \Prev\aformula \mid \aformula \Until \aformulabis \mid \aformula\Since\aformulabis \mid \Assign{\avar}{\aterm} \mid \Assign{\avar}{\adatum},
    \end{array}
\]
where: $\aterm,\aterm'$ are terms, $\avar\in\VAR$, and $\adatum\in\adomain$.  

 Other operators are defined as usual, in particular $\aterm\neq \aterm'\egdef \neg(\aterm = \aterm')$, $\Eventually\aformula \egdef \top\Until\aformula$ and $\Always\aformula \egdef \neg\Eventually\neg\aformula$. 


\paragraph{Semantics.} A \emph{concrete path} over $\adomain$ is an element $\apath\in\{f\mid f:\VAR\to\adomain\}^\omega$. By $\apath(i)(\avar)$ we denote the value assigned to the variable $\avar$ at position $i$ of the path.
Let $\apath$ be a path, and let $i\geq 0$, the \defstyle{satisfaction relation} $\models$ between $\apath$ and formulas is inductively defined as follows:
\[
\begin{array}{lcl}
\apath,i \models \aterm \sim \aterm' & \iffdef & \pos^i(\aterm) \geq 0, \ \pos^i(\aterm') \geq 0 \mbox{ and } \\ & & \apath(\pos^i(\aterm))(\aterm_v) \sim \apath(\pos^i(\aterm))(\aterm'_v) \ \  \mbox{ (for ${\sim}\in\{=,<\}$) } \\
\apath,i \models \aterm = \adatum & \iffdef & \pos^i(\aterm) \geq 0 \mbox{ and } \apath(\pos^i(\aterm))(\aterm_v) = \adatum \\
\apath,i \models \neg\aformula & \iffdef & \apath,i \nVdash\aformula \\
\apath,i \models \aformula\vee\aformulabis & \iffdef & \apath,i \models \aformula \mbox{ or } \apath,i \models \aformulabis \\
\apath,i \models \Next\aformula & \iffdef & \apath,i+1 \models \aformula \\ 
\apath,i \models \Prev\aformula & \iffdef & i>0 \mbox{ and } \apath,i-1 \models \aformula \\
\apath,i \models \aformula\Until\aformulabis & \iffdef & \apath,j \models \aformulabis \mbox{ for some } j\geq i \mbox{ s.t. } \apath,k \models \aformula \mbox{ for all } i\leq k < j \\
\apath,i \models \aformula\Since\aformulabis & \iffdef & \apath,j \models \aformulabis \mbox{ for some } 0\leq j \leq i \mbox{ s.t. } \apath,k \models \aformula \mbox{ for all } j< k \leq i \\
\apath,i \models \Assign{\avar}{\Nextn\avarbis}\aformula & \iffdef & \apath[\avar\mapsto_i \apath(i+n)(\avarbis)],i \models \aformula \\ 
\apath,i \models \Assign{\avar}{\Prevn\avarbis}\aformula & \iffdef & i-n\geq 0 \mbox{ and } \apath[\avar\mapsto_i \apath(i-n)(\avarbis)],i \models \aformula \\
\apath,i \models \Assign{\avar}{\adatum}\aformula & \iffdef & \apath[\avar\mapsto_i \adatum],i \models \aformula, \\ 
\end{array}
\]
where $\apath[\avar\mapsto_i \adatum](j)(\avarbis)=\apath(j)(\avarbis)$ if $j\neq i$  or $\avar\neq\avarbis$, and $\apath[\avar\mapsto_i \adatum](\avar)(i)  = \adatum$. 
We say a formula $\aformula$ is \defstyle{satisfiable} if and only if there exists $\apath$ such that $\apath,0\models\aformula$.






\section{$\PLTL$ with Propositional Assignments}

\input{definitions}

\section{The case of $\PLTLA(\natnum)$}

\begin{theorem}
 The satisfiability problem for $\PLTLA(\adomain)$ is \tower-hard.
\end{theorem}

\begin{proof}
Hardness follows from hardness from~\Cref{sec:prop-assign}.

    % \bigraul{The question is, can we reduce it into $\PLTLA(\adomain)$ with only 1 variable??}
\end{proof}

\subsection{Detour: A new decidable \FOL fragment over multi-data words} 

We need to introduce a new logic \defstyle{local \FOL over multi-data words}. Let $\{f_1,\dots,f_\beta\}$ be a set of function symbols:
\[
\begin{array}{l@{\;}l}
    \aformula, \aformulabis ::= & \avar = \avarbis \mid \avar < \avarbis \mid \avar = \avarbis + 1 \mid a(\avar) \mid \neg\aformula \mid \aformula \vee \aformulabis \\ 
    & \mid \exists x.\varphi \mid f_i(\avar + k_1) \sim f_j(\avar+k_2) \mid f_i(\avar + k_1) < f_j(\avar+k_2) \mid f_i(\avar + k_1) = a
\end{array}
\]

Models are of the form $(\adomain^\beta)^\omega$, and we call them multi-data words. \raul{This is just a sketch of the defs. We need to polish them}

\begin{theorem}
 The satisfiability problem for local \FOL over multi-data words is in \tower.
\end{theorem}

\begin{proof}
Lower bound follows from \tower-completeness of \FOL. \raul{Reference?}. Upper bound to check satisfiability of $\aformula$: 
\begin{enumerate}
    \item Use the notion of transfer graphs to characterize a window of comparison.
    \item Reduce the satisfiability problem into satisfiability over symbolic models (each letter of the path is a set of constraints over the variables involved).
    \item It remains to extract a concrete model from the symbolic model. This is performed due to \cite{Demri&Quaas2023}, by using the B\"chi automata $\mathbb{A}_\aformula$.
    \bigraul{Some missing step?}
\end{enumerate}
\end{proof}

\begin{conjecture}
    There is a translation from $\PLTLA(\adomain)$ into `local' $\FOL$ over multi-data words. Thus, the satisfiability problem for $\PLTLA(\adomain)$ is in \tower.
\end{conjecture}


\bibliographystyle{plain}
\bibliography{biblio}


\end{document}
