\documentclass[envcountsame,a4paper,12pt]{llncs}


\usepackage[T1]{fontenc}
%\usepackage[latin1]{inputenc}
\usepackage[x11names,dvipsnames]{xcolor}
\usepackage{amsmath}
\usepackage{amssymb}
\usepackage{verbatim}
\usepackage{booktabs}
\usepackage{multicol}
\usepackage{xspace}
\usepackage{underscore}
\usepackage{upgreek}
\usepackage[textsize=scriptsize]{todonotes}
\usepackage{lineno}
%\usepackage{pxfonts}
\usepackage{algorithm}
\usepackage{algpseudocode}
\usepackage{stmaryrd}

\usepackage[draft]{fixme}

\usepackage[font=small,skip=0pt]{caption}

\usepackage{mathabx}
\usepackage{wasysym}
%------------------------------------------------------------------------------------------------

%------------------------------------------------------------------------------------------------

\usepackage{hyperref}

\hypersetup{
  pdfstartview      = {FitH},
  pdfpagelayout     = {OneColumn},
  pdfpagelabels     = true,             % for the reader to display the page number as 'ii (4 of 40)' rather than simply '4 of 40'
  plainpages        = false,            % page anchors with formatted form of the page number (i.e., different anchors for pages 'ii' and '2')
  bookmarksnumbered = {true},
  naturalnames      = {true},
  colorlinks        = {true},           
  linkcolor         = blue,             
  anchorcolor       = red,              % ?
  urlcolor          = NavyBlue,         
  citecolor         = BrickRed          
}

%------------------------------------------------------------------------------------------------

\usepackage{macros}

%------------------------------------------------------------------------------------------------

\renewcommand{\arraystretch}{1.3}
\renewcommand{\emptyset}{\varnothing}
\newcommand{\keywords}[1]{\medskip\noindent\tb{Keywords}: #1}

%------------------------------------------------------------------------------------------------

% \title{Reasoning with Constrained Knowing How Logics}
%% Just another suggestion to stimulate us
%% \title{Model-checking for Knowing How Logics \\ with Constrained Plans}
\title{Note about  Decidability  of $\modelchecking{\WanglogicBudget{\star}}$}
\author{}

\date{ } 

%------------------------------------------------------------------------------------------------

\begin{document}
\pagestyle{plain}

%\linenumbers

\maketitle
%% \input{abstract}
%%

\section{Initial Decision Problem $\modelchecking{\WanglogicBudget{\star}}$}

Given a set $\states$ of states  and a finite set $\states$ of actions, a weight function is a map
$\wf : \states \times  \ACT \times \states \ra \Zed^r$ for some  $r{\geq}0$ (a bit more general than the definition
in~\cite{Demri&Fervari23}).
Let $\WanglogicBudget{r}$ be the ability-based logic $\Wanglogic$ augmented with $r \geq 0$ resource types.
The logic $\WanglogicBudget{\star}$ denotes the version in which the number of resource types is arbitrary.
Models of $\WanglogicBudget{r}$ are of the form $\alts = \triple{\states}{\relations,\wf}{\avaluation}$
\textcolor{blue}{S: better to use $L$ for Labellings and $V$ for Valuations.}
where $ \states$ is a non-empty set of states, $\relations$ is a collection of binary relations on $ \states$, 
$\wf: \states \times  \ACT \times \states \ra \Zed^r$ is a weight function
(the weights are of dimension $r$) and $\avaluation: \states \to 2^\PROP$ is a labelling function.
Given a computation 
$\acomputation = \astate_0 \step{\aaction_1} \astate_1 \step{\aaction_2} \astate_2 \cdots 
\step{\aaction_K} \astate_K$, its weight is defined as %the sum 
$\wf(\acomputation)  \egdef \Sigma^{K}_{k=1} \wf(\astate_{k-1},\aaction_k, \astate_{k})$ (empty computations have zero weight).
%%
Formulae of the logic $\WanglogicBudget{r}$ are defined by the grammar:
    \begin{nscenter}
      \ \ \ \ \ \ \ \ \ \ \ \ $\aformula ::= \avarprop \mid \neg \aformula \mid \aformula \vee \aformula  \mid
      \kh^{\vec{b}}(\aformula,\aformula)$ \hfill ($\avarprop \in \PROP$, $\vec{b} \in \Nat^r$).
    \end{nscenter}

The satisfaction relation $\models$ is defined as follows:

   \begin{nscenter}
     \begin{tabular}{@{}lcl@{}}
       $\modlts,\astate \models \avarprop$ & $\iffdef$ & $\avarprop \in\V(\astate)$, \\
       $\modlts,\astate \models \neg \aformula$ & $\iffdef$ & $\modlts,\astate \nVdash\aformula$, \\ 
       $\modlts,\astate \models \aformula \vee \aformulabis $ & $\iffdef$ & $\modlts,\astate \models \aformula 
       \,\mbox{ or }\, \modlts,\astate \models\aformulabis$, \\
       $\modlts,\astate \models \kh^{\vec{b}}(\aformula,\aformulabis)$ & $\iffdef$  & there exists $\aplan\in\ACT^*$ such that \\
                                    & & 
                                                            (1) $\truthset{\modlts}{\aformula }\subseteq\stexec(\aplan)$, \\
       & &                     (2) $\R_\aplan(\truthset{\modlts}{\aformula}) \subseteq \truthset{\modlts}{\aformulabis}$ and \\
       & &                     (3) $\aplan$ is $\vec{b}$-compatible at $\truthset{\model}{\aformula}$, 
     \end{tabular}
   \end{nscenter}
    with $\truthset{\modlts}{\aformulater} \egdef \csetc{\astate}{ \states}{\modlts,\astate\models \aformulater}$. 

    Let us recall a few definitions to understand the definition above.
    Given a plan $\aplan \in \ACT^*$,
    \begin{itemize}
    \item we write $\length{\aplan}$ to denote its length,
    \item for all $k \in \interval{0}{\length{\aplan}}$, $\aplan[0,k)$ is the prefix of
      $\aplan$ of length $k$,
    \item if $\length{\aplan} > 0$, $\aplan$ can be also written $\aplan[0] \cdots
      \aplan[\length{\aplan}-1]$. 
    \end{itemize}
    A plan $\aplan \in \ACT^*$ is 
    strongly executable (SE) at $\astate \in  \states$ 
    iff 
    for all $k \in \interval{0}{\length{\aplan}-1}$ and  $\astatebis \in \R_{\aplan[0,k)}(\astate)$, 
    we have $\R_{\aplan[k]}(\astatebis) \neq \emptyset$.
     We define the set $\stexec(\aplan) \egdef \cset{ \astate \in \states \mid \aplan \mbox{ is SE at }\astate}$.
     %% 
     A plan $\aplan = \aaction_1 \cdots \aaction_K$ is $\vec{b}$-compatible at $\astate$ ($\vec{b} \in \Nat^r$) $\equivdef$ for every computation
     $\acomputation = \astate_0 \step{\aaction_1} \astate_1 \cdots 
     \step{\aaction_K} \astate_K$ with $\astate_0 = \astate$, we have
     for all $L \in \interval{1}{K}$, $\vec{b} + \wf(\acomputation_{\leq L}) \geq \vec{0}$ (with
     $\acomputation_{\leq L} \egdef \astate_0 \step{\aaction_1} \astate_1 \cdots 
     \step{\aaction_L} \astate_L$). The plan $\aplan$ is $\vec{b}$-compatible at a set 
      $\aset \subseteq\states$ $\equivdef$ it is $\vec{b}$-compatible at all $\astate\in \aset$.

     $\modelchecking{\WanglogicBudget{\star}}$ is \expspace-hard~\cite{Demri&Fervari23}
     by reduction from the control state reachability problem for VASS, decidability is left open.
     %%
     This note is about the decidability status of $\modelchecking{\WanglogicBudget{\star}}$.
     

     \section{Variant problem on VASS}

     
A vector addition system with states (VASS)
  is a structure
  $\avass = \triple{\locations,\actions}{r}{R}$, where
  $\locations$ is a finite set of locations,
  $r{\in}\Nat$ is its dimension, 
  and $R$ is a finite set of transitions in $\locations \times \actions \times \Zed^r
  \times \locations$ (herein, the transitions are labelled by letters from the alphabet $\actions$).
  \textcolor{blue}{S: It is unclear it is better to introduce a restricted class with a sink location and every action can be always fired.}
  A configuration (resp. pseudo-configuration)  in a VASS $\avass$ is
a pair $\pair{\alocation}{\vec{x}} \in \locations \times \Nat^r$ (resp. in $\locations \times \Zed^r$).
Given pseudo-configurations $\pair{\alocation}{\vec{x}}$,
$\pair{\alocation'}{\vec{x}'}$ and a transition $\atransition =
\alocation \step{\aaction,\vec{u}} \alocation'$, we write
$\pair{\alocation}{\vec{x}} \step{\atransition}
\pair{\alocation'}{\vec{x}'}$ whenever $\vec{x'} = \vec{u} + \vec{x}$.
A pseudo-run is defined as a sequence $\arun =
\pair{\alocation_0}{\vec{x}_0} \step{\atransition_1}
\pair{\alocation_1}{\vec{x}_1} \step{\atransition_2}
\pair{\alocation_2}{\vec{x}_2} \cdots$ of pseudo-configurations, where
$\pair{\alocation_0}{\vec{x}_0}$ is the initial
pseudo-configuration. A run is defined as a pseudo-run in which 
only configurations in $\locations \times \Nat^r$ occur. 
An $r$-VASS is a VASS with $r \geq 0$
counters. 
The label of a pseudo-run $\arun =
\pair{\alocation_0}{\vec{x}_0} \step{\atransition_1}
\pair{\alocation_1}{\vec{x}_1} \step{\atransition_2}
\pair{\alocation_2}{\vec{x}_2} \cdots$ with
$\atransition_i = 
\alocation_i \step{\aaction_i,\vec{u_i}} \alocation'_i$
is the sequence $\aaction_1 \aaction_2 \cdots$. 

In order to show the decidability of  $\modelchecking{\WanglogicBudget{\star}}$, we reduce it to the
problem \ourvasspb defined below.

\begin{description}
  \item[Input:] A VASS $\avass = \triple{\locations,\actions}{r}{R}$,
    $\alocationbis_0 \in \locations$, $\vec{y_0} \in \Nat^r$, $\locations_F \subseteq \locations$.  

  \item[Question] Is there $\aplan \in \actions^*$  such that
    for all finite pseudo-runs
    $\arun =
\pair{\alocation_0}{\vec{x}_0} \step{\atransition_1}
\pair{\alocation_1}{\vec{x}_1} \step{\atransition_2}
\pair{\alocation_2}{\vec{x}_2} \cdots
\step{\atransition_L}
\pair{\alocation_L}{\vec{x}_L}$
from $\pair{\alocationbis_0}{\vec{y_0}}$ with label $\aplan$,
we have $\alocation_L \in \locations_F$ and $\set{\vec{x}_0,\ldots,\vec{x}_L}
\subseteq \Nat^r$ (hence $\arun$ is a run)?
\end{description}

\textcolor{blue}{S: To compare this problem with non-termination problem in~\cite{Perez17}.}

\begin{lemma} \label{lemma-from-mc-to-p}
  There is a Turing reduction from $\modelchecking{\WanglogicBudget{\star}}$ to
  \ourvasspb. 
\end{lemma}

As a consequence, the decidability of  \ourvasspb would entail the decidability of
$\modelchecking{\WanglogicBudget{\star}}$. Moreover, the exact definition of the Turing reduction
might help to get a tight complexity upper bound.

\input{proof-lemma-from-mc-to-p}



\begin{lemma} There is a logarithmic space many-one reduction
 from \ourvasspb to its restriction of dimension one.
\end{lemma}

\begin{proof} (sketch)
Let $\avass = \triple{\locations,\actions}{r}{R}$,
$\alocationbis_0 \in \locations$, $\vec{y_0} \in \Nat^r$ and $\locations_F \subseteq \locations$
be an instance of \ourvasspb. Without any loss of generality, we can assume that
$\alocationbis_0 \not \in \locations_F$, otherwise the empty plan is a solution.

Let us define the one-VASS $\avass' = \triple{\locations',\actions}{1}{R'}$ as follows:
\begin{itemize}
\item $\locations' \egdef \set{\alocationbis_0} \uplus \locations \times \interval{1}{r}$.
\item $R'$ is the (disjoint) union of the following sets of transitions ($\aaction^{\dag} \in \actions$ is a
  distinguished action)
  \begin{itemize}
  \item $\set{\triple{\alocationbis_0}{\aaction^{\dag},\vec{y_0}[i]}{\pair{\alocationbis_0}{i}} \mid 
    i \in \interval{1}{r}}$.
  \item $\set{\triple{\pair{\alocation}{i}}{\aaction,\vec{u}[i]}{\pair{\alocation'}{i}} \mid
    i \in \interval{1}{r}, \triple{\alocation}{\aaction, \vec{u}}{\alocation'} \in R}$.
    Transitions in the $i$th copy with states in $\locations \times \set{i}$
    take care of the updates in the $i$th component of the updates in $\Zed^r$. 
  \end{itemize}
\end{itemize}
One can show that
$\avass$,
$\alocationbis_0$, $\vec{y_0}$, $\locations_F$
is a positive instance of \ourvasspb
iff
$\avass'$, $\alocationbis_0$, $0$, $\locations_F \times \interval{1}{r}$
is a positive instance of \ourvasspb.

First, suppose that $\avass$,
$\alocationbis_0$, $\vec{y_0}$, $\locations_F$
is a positive instance of \ourvasspb.
There is a plan $\aplan \in \actions^*$  such that
    all finite pseudo-runs
    $\pair{\alocation_0}{\vec{x}_0} \step{\atransition_1}
\pair{\alocation_1}{\vec{x}_1} \step{\atransition_2}
\pair{\alocation_2}{\vec{x}_2} \cdots
\step{\atransition_L}
\pair{\alocation_L}{\vec{x}_L}$
from $\pair{\alocationbis_0}{\vec{y_0}}$ with label $\aplan$,
we have $\alocation_L \in \locations_F$ and $\set{\vec{x}_0,\ldots,\vec{x}_L}
\subseteq \Nat^r$.

Let $\arun = \pair{\alocationbis_0}{0} \step{\atransition_0} \pair{\pair{\alocation_0}{i}}{z_0}
\step{\atransition_1}
\pair{\pair{\alocation_1}{i}}{z_1} \step{\atransition_2}
\pair{\pair{\alocation_2}{i}}{z_2} \cdots
\step{\atransition_L}
\pair{\pair{\alocation_L}{i}}{z_L}$
be a pseudo-run from $\pair{\alocationbis_0}{0}$
with label $\aaction^{\dag} \cdot \aplan$ (for some $i \in \interval{1}{r}$).
By definition of $\avass'$, we have $z_0 = \vec{y_0}[i]$.
Moreover, by construction of $\avass'$ from $\avass$,
there is a pseudo-run
$\pair{\alocation_0}{\vec{x}_0} \step{\atransition_1'}
\pair{\alocation_1}{\vec{x}_1} \step{\atransition_2'}
\pair{\alocation_2}{\vec{x}_2} \cdots
\step{\atransition_L'}
\pair{\alocation_L}{\vec{x}_L}$
such that for all $j \in \interval{0}{L}$, we have $\vec{x}_j[i] = z_i$.
Since $\aplan$ is a solution for $\avass$, $\alocation_L \in \locations_F$
and $\set{z_0,\ldots,z_L} \subseteq \Nat$.
Consequently, $\aaction^{\dag} \cdot \aplan$ is a solution for the instance with $\avass'$. 

Conversely, suppose that $\avass'$, $\alocationbis_0$, $0$, $\locations_F \times \interval{1}{r}$
is a positive instance of \ourvasspb.
There is a plan $\aaction^{\dag} \cdot \aplan \in \actions^*$  such that
all pseudo-runs
$\pair{\alocationbis_0}{0} \step{\atransition_0} \pair{\pair{\alocation_0}{i}}{z_0}
\step{\atransition_1}
\pair{\pair{\alocation_1}{i}}{z_1} \step{\atransition_2}
\pair{\pair{\alocation_2}{i}}{z_2} \cdots
\step{\atransition_L}
\pair{\pair{\alocation_L}{i}}{z_L}$
with label $\aaction^{\dag} \cdot \aplan$,
we have $\alocation_L \in \locations_F$ and
 $\set{z_0,\ldots,z_L} \subseteq \Nat$.
Let 
$\arun = \pair{\alocation_0}{\vec{x}_0} \step{\atransition_1}
\pair{\alocation_1}{\vec{x}_1} \step{\atransition_2}
\pair{\alocation_2}{\vec{x}_2} \cdots
\step{\atransition_L}
\pair{\alocation_L}{\vec{x}_L}$
be a run from $\pair{\alocationbis_0}{\vec{y_0}}$ with label $\aplan$.
Since $\aaction^{\dag} \cdot \aplan$ is a solution for the instance with $\avass'$,
for all $i \in \interval{1}{r}$, for all pseudo-runs 
$\pair{\alocationbis_0}{0} \step{\atransition_0} \pair{\pair{\alocation_0}{i}}{\vec{x}_0[i]}
\step{\atransition_1}
\pair{\pair{\alocation_1}{i}}{\vec{x}_1[i]} \step{\atransition_2}
\pair{\pair{\alocation_2}{i}}{\vec{x}_2[i]} \cdots
\step{\atransition_L}
\pair{\pair{\alocation_L}{i}}{\vec{x}_L[i]}$
with label $\aaction^{\dag} \cdot \aplan$,
we have $\alocation_L \in \locations_f$ and
$\set{\vec{x}_0[i],\ldots,\vec{x}_L[i]} \subseteq \Nat$.
Consequently $\alocation_L \in \locations_F$ and $\set{\vec{x}_0,\ldots,\vec{x}_L}
\subseteq \Nat^r$.
Hence, $\avass$,
$\alocationbis_0$, $\vec{y_0}$, $\locations_F$
is a positive instance of \ourvasspb.
\end{proof}
%%% AAAA 
%%
%% S 03/10/23 -- With news results, this section is of no use,
%% it seems to me. 
%%
\cut{
\input{section-digression}
}

\section{Definitions and decidability proof}

In this section, we provide material to show decidability of \ourvasspb with VASS of dimension one.
Let $\avass = \triple{\locations,\actions}{1}{R}$,
$\alocationbis_0 \in \locations$, $y_0 \in \Nat$, $\locations_F \subseteq \locations$ be an
instance of \ourvasspb.


A \defstyle{configuration} is a pair $\pair{\alocation}{v} \in \locations \times \Nat$.
A \defstyle{state} is a (possibly infinite) subset of $\locations \times \Nat$, i.e. a set of configurations.
Configurations in $\locations \times \Nat$ are equipped with the binary relation
$\preceq$ such that $\pair{\alocation}{v} \preceq \pair{\alocation'}{v'}$ $\equivdef$
$\alocation = \alocation'$ and $v \leq v'$.
We write $\uparrow_c \pair{\alocation}{v}$ to denote the set
$\set{\pair{\alocation'}{v'} \in \locations \times \Nat
  \mid \pair{\alocation}{v} \preceq \pair{\alocation'}{v'}}$
(upward closure of $\pair{\alocation}{v}$).
The subscript `c' refers to the fact that the upward closure is relative to a configuration.

  States in $\powerset{\locations \times \Nat}$ are also equipped with a binary relation
  $\preceq$ (overloaded symbol here) such that
  $\aset \preceq \asetbis$ $\equivdef$
  $$
  \asetbis
  \subseteq \bigcup_{\pair{\alocation}{v} \in \aset} \uparrow_c \pair{\alocation}{v},
  $$
  which is equivalent to
  $$
  \bigcup_{\pair{\alocation}{v} \in \asetbis}  \uparrow_c \pair{\alocation}{v}
  \subseteq \bigcup_{\pair{\alocation}{v} \in \aset} \uparrow_c \pair{\alocation}{v}.
  $$
  Similarly, $\aset \preceq \asetbis$ iff for all
  $\pair{\alocation}{v} \in \asetbis$, there is
  $\pair{\alocation}{v'} \in \aset$ such that $v' \leq v$. 

  We write $\uparrow_s \aset$ to denote the set
  $\set{\asetbis \in \powerset{\locations \times \Nat} \mid \aset \preceq \asetbis}$.
  The subscript `s' refers to the fact that the upward closure is relative to a state.
  Similarly, given a set $\mathcal{X} \subseteq \powerset{\locations \times \Nat}$ of states,
  we write $\uparrow_s \mathcal{X}$ to denote the set $\bigcup_{\aset \in \mathcal{X}} \uparrow_s \aset$.
  Hence, given $\aset \in \powerset{\locations \times \Nat}$,
  $\uparrow_s \aset$ and $\uparrow_s \set{\aset}$ are equal by definition.  
  $\mathcal{X}$ is upward closed whenever $\mathcal{X} = \uparrow_s \mathcal{X}$. 

  \begin{lemma}\label{lemma-wqo}
    The relation $\preceq$ on $\powerset{\locations \times \Nat}$ is a w.q.o.
  \end{lemma}

  \textcolor{blue}{S:We could also write down a direct proof or ask PhS for another adequate bibliographical
    reference.}

  \begin{proof} (sketch) As far as we can judge, a simple argument consists in observing
    that $\pair{\locations \times \Nat}{\preceq}$ does not contain an isomorphic copy of
    Rado structure and to apply~\cite[Theorem 1]{Jancar99} to get that
    $\preceq$ on $\powerset{\locations \times \Nat}$ is a w.q.o.
  \end{proof}


  Let us define the transition system $\pair{\powerset{\locations \times \Nat}}{\relations}$
  such that $\aset_1 \step{\aaction} \aset_2$ $\equivdef$
  \begin{enumerate}
  %% \item $\aset_2 \neq \emptyset$,
  \item for all $\pair{\alocation}{v} \in \aset_1$ and
    $\alocation \step{\aaction,u^{\star}} \alocation^{\star} \in R$, we have
    $v + u^{\star} \geq 0$ (``all the $\aaction$-transitions are fireable from all the configurations
    in $\aset_1$''),
  \item 
  $
  \aset_2 =
  \{
    \pair{\alocation'}{v'} \in \locations \times \Nat \mid
    \exists \ \pair{\alocation}{v} \in \aset_1 \
    \mbox{and}  \ 
    \alocation \step{\aaction,u} \alocation' \in R
    \ \mbox{s.t.} \ v' = v + u 
    \}$.
    \end{enumerate}
    Observe that each relation $\step{\aaction}$ is deterministic.
    Moreover, the definition of $\step{\aaction}$ above differs slightly from what is written
    in the notes in order to satisfy Lemma~\ref{lemma-monotony} below.
    We write $\pre{\aaction}{\mathcal{X}}$ to denote
    the set $\set{\aset' \mid \exists \ \aset \in \mathcal{X} \ \mbox{s.t.} \ 
      \aset' \step{\aaction} \aset}$ and
    $\pre{}{\mathcal{X}}$ to denote
    $\bigcup_{\aaction \in \actions} \pre{\aaction}{\mathcal{X}}$.
    The set $\prestar{\mathcal{X}}$ of states  is defined similarly.


    

  \begin{lemma}[Monotony]\label{lemma-monotony}
    If $\aset_1 \step{\aaction} \aset_2$ and $\aset_1 \preceq \aset_1'$, then
    there is $\aset_2 \preceq \aset_2'$ such that $\aset_1' \step{\aaction} \aset_2'$.
  \end{lemma}

  \begin{proof} Suppose that $\aset_1 \step{\aaction} \aset_2$ and $\aset_1 \preceq \aset_1'$.
    We write $\aset_2'$ to denote the set
    $
  \aset_2' =
  \{
    \pair{\alocation'}{v'} \in \locations \times \Nat \mid
    \exists \ \pair{\alocation}{v} \in \aset_1' \
    \mbox{and}  \ 
    \alocation \step{\aaction,u} \alocation' \in R
    \ \mbox{s.t.} \ v' = v + u 
    \}$.
    By construction, the condition (2.) is satisfied in order to guarantee
    $\aset_1' \step{\aaction} \aset_2'$. In order to check (1.),
    let $\pair{\alocation}{v} \in \aset_1'$ and
    $\alocation \step{\aaction,u^{\star}} \alocation^{\star} \in R$.
    Since $\aset_1 \preceq \aset_1'$, there is $\pair{\alocation}{v'} \in \aset_1$
    such that $v' \leq v$. Since (1.) holds for $\aset_1 \step{\aaction} \aset_2$,
    $v' + u^{\star} \geq 0$ and a fortiori, $v + u^{\star} \geq 0$.
    Consequently, $\aset_1' \step{\aaction} \aset_2'$.

    It remains to check that $\aset_2 \preceq \aset_2'$.
    Let $\pair{\alocation}{v} \in \aset_2'$. By definition of
    $\aset_2'$, there is $\pair{\alocation'}{v'} \in \aset_1'$
    and 
    $\alocation' \step{\aaction,u} \alocation \in R$
    such that $v = v' + u$.
    Since $\aset_1 \preceq \aset_1'$,
    there is $\pair{\alocation'}{v''} \in \aset_1$ such that
    $v'' \leq v'$.
    Since $\aset_1 \step{\aaction} \aset_2$,
    $\pair{\alocation}{v''+u} \in \aset_2$ and therefore
    $v'' +u \leq v'+u$, i.e. $\pair{\alocation}{v''+u} \preceq
    \pair{\alocation}{v'+u} = \pair{\alocation}{v}$. Consequently,
    $\aset_2 \preceq \aset_2'$. \qed 
    %% previous version
   \cut{
    Let $\aset_2'$ be the set of configurations
    $\pair{\alocation}{v}$ such that
    there are
    $\pair{\alocation'}{v''} \in \aset_1$,
    $\pair{\alocation'}{v'} \in \aset_1'$ with
    $$
    v'' \leq v', \ \ \ 
    \pair{\alocation'}{v''}
    \step{\aaction,u} \pair{\alocation}{v''+u}, \ \ \
    v = v'+u.
    $$
    It is easy to show that $\aset_2 \preceq \aset_2'$ since
    $v'' \leq v'$.
    Moreover, since $\aset_1 \step{\aaction} \aset_2$,
    we can conclude that $\aset_1' \step{\aaction} \aset_2'$.
    Indeed, all the $\aaction$-transitions are fireable
    from $\aset_1$ and a fortiori with greater values,
    all the $\aaction$-transitions are also  fireable
    from $\aset_1'$ heading to $\aset_2'$.
    }
  \end{proof}

  Lemma~\ref{lemma-monotony} allows us to conclude that
  $\pair{\powerset{\locations \times \Nat}}{\relations}$ is a
  well-structured transition systems (WSTS) in the sense of~\cite[Definition 2.5]{Finkel&Schnoebelen01}
  and therefore, in the sequel, we can take advantage of~\cite[Theorem 3.6]{Finkel&Schnoebelen01}. 

  There is a simple property (consequence of Lemma~\ref{lemma-monotony})
  that is used several times below. Let us make a lemma for it (folklore result at least).

  \begin{lemma}\label{lemma-folklore}
    Let $\mathcal{X}$ be an upward closed set of states.
    Then $\uparrow_s \prestar{\mathcal{X}} =  \prestar{\mathcal{X}}$. 
    \end{lemma}

  Before going any further, one can show the equivalence between the statements below.
  \begin{description}
  \item[(A)] $\avass = \triple{\locations,\actions}{1}{R}$,
    $\alocationbis_0 \in \locations$, $y_0 \in \Nat$, $\locations_F \subseteq \locations$
    is a positive instance of \ourvasspb. 
  \item[(B)] There is a path $\aset_0 \step{\aaction_1} \aset_1 \cdots \step{\aaction_{K}} \aset_K$
    in $\pair{\powerset{\locations \times \Nat}}{\relations}$
    with $\aset_0 = \set{\pair{\alocationbis_0}{y_0}}$
    and $\aset_K \subseteq \locations_F \times \Nat$.     
  \item[(C)] $\set{\pair{\alocationbis_0}{y_0}} \in \prestar{\uparrow_{s} \locations_F \times \set{0}}$
    ($\prestar{\uparrow_{s} \locations_F \times \set{0}}$ could be also written
     $\prestar{\uparrow_{s} \set{\locations_F \times \set{0}}}$).
  \item[(D)]
    $\set{\pair{\alocationbis_0}{y_0}} \in \uparrow_{s}
    \prestar{\uparrow_{s} \locations_F \times \set{0}}$.
  \end{description}
  The equivalence between (A) and (B) is by definition of
  $\pair{\powerset{\locations \times \Nat}}{\relations}$ and by using the fact that
  the action $\aaction$ is fireable from $\aset$ only if for all the configurations in $\aset$
  and for all $\aaction$-transitions in $\avass$ (with the same control state), the transition is
  fireable (with the new definition).
  The equivalence between (B) and (C) is just by taking advantage of the definition
  of $pre^{\star}$ and the fact that $\uparrow_{s} \locations_F \times \set{0}$
  contains all the states that contain only configurations with control states in $\locations_F$.
  The equivalence between (C) and (D), and in particular the implication from (C) to (D) is by 
  Lemma~\ref{lemma-folklore}. 

  The equivalence between (A) and (C) justifies why we can restrict ourselves to solve
  (C), which can be reformulated as a covering problem, see e.g.~\cite{Finkel&Schnoebelen01}.
  In order to take advantage of~\cite[Theorem 3.6]{Finkel&Schnoebelen01}, we need to explain how to represent
  upward closed sets $\mathcal{X}$.

  An \defstyle{abstract state skeleton} is a map $\amap: \locations \to \Nat \cup \set{\infty}$
  \textcolor{blue}{S: a better name to be found}.
  Given $\aset \subseteq \locations \times \Nat$, we define its abstract state skeleton $\amap_{\aset}$
  as follows: for all $\alocation \in \locations$,
  $$
  \amap_{\aset}(\alocation) \egdef
  \min (\set{\infty} \cup \set{v \mid \pair{\alocation}{v} \in \aset}). 
  $$
  Observe that $\alocation$ does not occur in $\aset$ iff $\amap_{\aset}(\alocation) = \infty$.
  Objects similar to abstract state skeletons have been introduced
  in~\cite[Section 3]{Degorreetal10},~\cite[Section 3]{Perez17} and~\cite[Section 5]{Hofman&Totzke18}.


  
  Given two abstract state skeletons $\amap$ and $\amapbis$, we define the binary relation
  $\preceq$ (again, the symbol is overloaded) such that
  $\amap \preceq \amapbis$ $\equivdef$ for all
  $\alocation \in \locations$, we have $\amap(\alocation) \leq \amapbis(\alocation)$, assuming
  that $\infty$ dominates all the natural numbers.
  The relation $\preceq$ on abstract state skeletons behaves very much like the component-wise less than
  relation on $\Nat^{\card{\locations}}$. 
  The property below can be easily shown.

  \begin{lemma} \label{lemma-good-abstraction}
    $\aset \preceq \asetbis$ iff
    $\amap_{\aset} \preceq \amap_{\asetbis}$. 
  \end{lemma}

  \begin{proof} \textcolor{blue}{TBC}
  \end{proof}

  Given $\amap: \locations \to \Nat \cup \set{\infty}$,
  we write $\uparrow \amap$ to denote the
  set of states $\set{\aset \subseteq \locations \times \Nat \mid \amap \preceq \amap_{\aset}}$
  (upward closed set by definition).
  Similarly, given a finite set of abstract state skeletons $\mathcal{Z}$,
  we write $\uparrow \mathcal{Z}$ to denote the set $\bigcup_{\amap \in \mathcal{Z}} \uparrow \amap$ of states
   (that is also a upward closed set by definition).

  \begin{lemma} \label{lemma-finite-representation}
    For every upward closed set $\mathcal{X} \subseteq \powerset{\locations \times \Nat}$,
    there is a finite set of abstract state skeletons $\mathcal{Z}$ such that
     $\uparrow \mathcal{Z} = \mathcal{X}$. 
  \end{lemma}

  Lemma~\ref{lemma-finite-representation} states that finite sets of abstract state skeletons
  provide a finite representation of
  upward closed subsets of $\powerset{\locations \times \Nat}$. 
  Its proof is mainly based on the fact that $\preceq$ on $\powerset{\locations \times \Nat}$
  is a w.q.o (see Lemma~\ref{lemma-wqo}).
  For instance, $\uparrow_s \set{\pair{\alocationbis_0}{y_0}} = \uparrow \set{\amap_0}$,
  where $\amap_0$ takes the value $\infty$ for all locations, except for $\alocationbis_0$
  with $\amap_0(\alocationbis_0) = y_0$. Similarly,
  $$
  \uparrow_{s} \locations_F \times \set{0} = \uparrow \set{\amap_F},
  $$
  where $\amap_F$ takes value $\infty$ for all locations in $\locations \setminus \locations_F$,
  and the value $0$ otherwise.

  \begin{proof} (sketch)
  Assume that $\mathcal{X}$ is upward closed.
  Since $\preceq$ on $\powerset{\locations \times \Nat}$
  is a w.q.o., there is a finite set $\mathcal{X}'$ (basis) such that
  $\uparrow_s \mathcal{X}' = \mathcal{X}$ (see e.g.~\cite[Theorem 2.3]{Finkel&Schnoebelen01}).
  Let $\mathcal{Z}$ be equal to $\set{\amap_{\aset} \mid \aset \in \mathcal{X}'}$.
  One can show that $\uparrow \mathcal{Z} = \mathcal{X}$.

  Indeed, $\aset \in \mathcal{X}$ implies there is $\aset' \in \mathcal{X'}$ such that
  $\aset' \preceq \aset$. By Lemma~\ref{lemma-good-abstraction},
  $\amap_{\aset'} \preceq \amap_{\aset}$. Consequently,
  $\aset \in \uparrow \amap_{\aset'}$ and therefore $\aset \in \uparrow \mathcal{Z}$
  since $\uparrow \mathcal{Z} = \bigcup_{\amap \in \mathcal{Z}} \uparrow \amap$.

  Conversely, let $\aset \in \uparrow \mathcal{Z}$, i.e. there is $\amap \in \mathcal{Z}$
  such that $\amap \preceq \amap_{\aset}$. By definition of $\mathcal{Z}$, there is $\aset'
  \in \mathcal{X'}$ such that $\amap = \amap_{\aset'}$ and therefore by
  Lemma~\ref{lemma-good-abstraction}, we get $\aset' \preceq \aset$.
  Since $\mathcal{X}$ is upward closed, we get $\aset \in \mathcal{X}$.\qed
  \end{proof}

  

  Given $\amap: \locations \to \Nat \cup \set{\infty}$ and
  $\aaction \in \actions$, we define the abstract state skeleton
  $pb(\aaction, \amap):  \locations \to \Nat \cup \set{\infty}$ as follows.
  Note that `pb' stands for ``pre basis'' (\textcolor{blue}{find another name}). 
  Given $\alocation \in \locations$,
  $pb(\aaction, \amap)(\alocation)$ is equal to $\infty$ if
  there is $\alocation \step{\aaction,u} \alocation' \in R$ such that
  $\amap(\alocation') = \infty$ or there is no transition in $R$ of the form
  $\alocation \step{\aaction,u} \alocation'$.
  Otherwise,
  $$
  pb(\aaction, \amap)(\alocation)
  \egdef
  \max
  (\set{ \amap(\alocation') - u \mid \alocation \step{\aaction,u} \alocation' \in R} \cup \set{0}). 
  $$

  Below, we present the main technical lemma to get decidability. 
  
  \begin{lemma} \label{lemma-pb}
  $\uparrow pb(\aaction, \amap)  = \uparrow_{s} \pre{\aaction}{\uparrow \amap}$. 
  \end{lemma}

  \input{proof-lemma-pb}

   As a consequence, for all finite sets of abstract state skeletons $\mathcal{Z}$, we have
  $\uparrow \set{pb(\aaction, \amap) \mid \amap \in  \mathcal{Z}}  = \uparrow_{s} \pre{\aaction}{\uparrow \mathcal{Z}}$ and therefore
  $$
  \uparrow \set{pb(\aaction, \amap) \mid \amap \in  \mathcal{Z}, \aaction \in \actions}  =
  \uparrow_{s} \prestar{\uparrow \mathcal{Z}}.
  $$
  
  In order to solve (D), we proceed as follows using the proof method developped
  in~\cite{Finkel&Schnoebelen01}. We define a family $(\mathcal{Z}_i)_{i \in \Nat}$
  made of finite sets of abstract state skeletons such that
  \begin{itemize}
  \item $\mathcal{Z}_{0} \egdef \set {\amap_F}$,
  \item $\mathcal{Z}_{i+1} \egdef \mathcal{Z}_{i}
    \cup \set{pb(\aaction,\amap) \mid \aaction \in \actions, \
      \amap \in \mathcal{Z}_{i}}$.
  \end{itemize}
  By~\cite[Lemma 2.4]{Finkel&Schnoebelen01}, there is $m$ such that
  $\uparrow \mathcal{Z}_{m} = \uparrow \mathcal{Z}_{m+1}$ and therefore
  the family $(\uparrow \mathcal{Z}_i)_{i \in \Nat}$ stabilizes at some point.
  One can show that (D) holds iff $\uparrow \amap_0 \subseteq \uparrow \mathcal{Z}_{m}$ (proof to be
  provided at some point based on $\uparrow \mathcal{Z}_{m} = \prestar{\uparrow_{s} \locations_F \times \set{0}}$).
  Since computing $pb(\aaction,\amap)$ is effective and
  checking whether  $\uparrow \mathcal{Z} \subseteq \uparrow \mathcal{Z}'$ can be done effectively too,
  this provides a decision procedure for (D).
  For instance,
  $\uparrow \mathcal{Z} \subseteq  \uparrow \mathcal{Z}'$ iff
  for all $\amap \in \mathcal{Z}$, there is  $\amap' \in \mathcal{Z}'$
  such that $\amap' \preceq \amap$
  \textcolor{blue}{(proof to be written down)}. 
  
  Based on the previous developments, decidability can be concluded. 
  
  \begin{theorem} $\modelchecking{\WanglogicBudget{\star}}$ is decidable. 
  \end{theorem}

  %% S 03/10/23 -- not anymore useful
  \cut{
  It is worth exploring whether  an \expspace upper bound could be obtained along the lines
  of~\cite{Bozzelli&Ganty11,Lazic&Schmitz21}. After all, the $\mathcal{Z}$'s are almost finite sets of tuples
  from some $\Nat^k$. Documents to read if we go to ACK-hardness include ~\cite{Degorreetal10,Perez17}. 
  }
 

\bibliographystyle{plain}
\bibliography{biblio}


%------------------------------------------------------------------------------------------------
%\clearpage
%\appendix
%\input{appendix}
\cut{
\appendix
\newpage 
\input{note1}
}
\end{document}
